\chapter{Introdução}


\setcounter{page}{1}    % set page to 1 again to start arabic count
\pagenumbering{arabic}


%Capítulo 1 de Szwarcfiter, \textit{Grafos e Algoritmos Computacionais}~\cite{Szwarcfiter1986grafos}.

%%%%%%%%%%%%%%%%%%%%%%%%%%%%%%%%%%%%%%%%%%%%%%%%%%%%%%%%%%%%
\section{Exercícios preliminares}

\begin{enumerate}
\item Defina matriz quadrada.
\item Defina matriz simétrica.
\item O que é um conjunto?
\item Defina cardinalidade de um conjunto.
\item Defina par ordenado.
\item Defina par não-ordenado.
\item Defina produto cartesiano de dois cojuntos.
\item Defina conjunto das partes.
\item Qual é a cardinalidade do conjunto das partes de um conjunto $A$ em função da cardinalidade de $A$?
\item Quando podemos dizer que dois conjuntos são iguais?
\item A cardinalidade do conjunto dos números naturais é maior, menor ou igual à do conjunto dos números racionais?
\item A cardinalidade do conjunto dos números naturais é maior, menor ou igual à do conjunto dos números reais?
\item Como provar que dois conjuntos infinitos possuem a mesma cardinalidade?
\item Defina função bijetora.
\item Seja $R$ o conjunto de todos os conjuntos que não pertencem a si mesmos. Podemos dizer que $R$ pertence a si mesmo?
\end{enumerate}

\begin{easylist}

\end{easylist}
